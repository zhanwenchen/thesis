\chapter{Conclusion}
The main contribution of this thesis is the study of convolutional neural networks in suppressing off-axis scattering noise in ultrasound beamforming. I showed that models with convolutional layers can improve the CNR in beamformed images by denoising the channel data in the STFT domain relative to the benchmark DAS beamformed images. Models with both convolutional and fully-connected layers are able to match the performance of MLPs with the additional benefit of having fewer total weights in the network. However, models with only convolutional layers do not perform as well as MLPs. I studied the effect on CNR of the convolutional kernel size and number of convolutional layers and determined that convolution does not contribute to learning any more than it is approximating full connections by having a large enough receptive field to cover either the complex component or the entire input space.

Opportunities for future work include the creation a new unified training and evaluation dataset from physical phantoms to solve the training and test domain mismatch problem inherent in our study. Another direction is to develop a differentiable loss function from our model selection metric in the beamformed image domain - the CNR, for which I have contributed some initial work. This will solve the learning objectivce mismatch problem also previously described.
