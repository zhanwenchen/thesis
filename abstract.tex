% (This is included by thesis.tex; you do not latex it by itself.)

\begin{abstract}

  Medical ultrasound is a noninvasive, affordable, portable, and real-time diagnostic modality that provides cross-sectional views of human tissues. Ultrasound beamforming is a widely used approach to acquire and process data from ultrasound probes in a focused manner. However, noises from tissue layering cause artifacts such as off-axis scattering and reverberation clutter and degrade the beamformed images. Recently, deep neural networks have been applied to medical imaging. In particular, Luchies et al. proposed multi-layer perceptrons (MLPs) operating on the frequency domain that prove effective in suppressing off-axis scattering and increasing contrast in beamformed images. This thesis extends the frequency-domain neural network approach to study the effectiveness of convolutional neural networks (CNNs). A variety of convolutional architectures are proposed, along with a constraint-satisfaction framework to conduct random hyperparameter search that solves for random architecture layer sizes given different input and output sizes. Various training hyperparameters are also used to investigate their effects on image quality. Preliminary results show that CNNs can achieve a similar level of performance for suppressing off-axis scattering and increasing contrast.

\end{abstract}
