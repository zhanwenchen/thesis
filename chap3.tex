\chapter{Methods}


The MLP beamformer studied here was proposed previously by members of our group\cite{luchies_tmi_2018}. A collection of MLPs is trained to operate on channel data in the frequency domain. Training data was generated from Field II simulated point target responses \cite{field_ii}. The simulated ultrasonic array was based on the L7-4 (38 mm) linear array transducer. Point targets were randomly placed in an annular sector centered at the focal depth of the transducer array, using a process that we have described previously \cite{training_improvements}. For the point targets inside the main lobe of the beam, the corresponding output was the same as the input; for those outside the main lobe of the beam, the corresponding output was a vector of zeros [cite]. We used 105 examples for training and 104 for validation.

% Figure 3 Scatterers were randomly placed along the annular sector as depicted. The acceptance region was taken as the region between the first nulls of a simulated beam. Question: Cite this figure? Not mine.


% TODO: Move to methods.
\subsection{STFT/Frequency Domain}

Frequency-domain, not aperture domain. Minimum-variance by f. grant and jorgen jansen in frequency domain.

STFT Domain. 1. Resilience changes in pulse shape. 2. we can do it in the complex data instead of hilbert (separation of components).

Aperture domain is about channel data.
