\section{Resolution in Ultrasound Imaging}

"Image resolution is much more than the ability so see small details. A good image provides information about tissue characteristics (i.e. soft tissue vs. fibrosis), has a low signal-to-noise ratio, is free of artifacts, has equal image properties throughout the imaging sector, and permits us to track the motion of structures.

To describe the "quality" of a 2D image one would have to consider several aspects:

Spatial or detail resolution is the ability to distinguish between distinct image points (reflectors) lying close to each other.
Lateral resolution describes the minimum separation of two reflectors aligned along a direction perpendicular to the ultrasound beam. Lateral resolution depends on beam width and scan line density.
Axial resolution is the minimum separation of two reflectors aligned along a direction perpendicular to the ultrasound beam. Axial resolution is influenced by pulse length and transducer frequency.
Contrast resolution is the ability to identify differences in echogenicity between adjacent soft tissue regions.
Sensitivity is the ability to visualize weakly reflective signals and distinguish them from noise (signal-to-noise ratio)
Temporal resolution refers to the ability to visualize moving objects. The main determinant of temporal resolution is the frame rate.
The components of image resolution are strongly related to the physical principles of ultrasound and transducer technology (such as imaging frequency, beam width, scan line density). However, they can also be altered by the manner in which the returning signal is recorded and digitized (preprocessing). Preprocessing techniques include dynamic range manipulation, gain/attenuation adjustments, and depth compensation. The digitized signal can also be altered (postprocessing). Commonly used techniques are compounding (averaging several frames) and spatial interpolation. Some parameters can also be adjusted by the user, such as grayscale representation or contrast, color maps, or filters that "smooth" the image or enhance contours.

One is always confronted with a trade-off between the various components of image quality. For example, the wider the sector is, the poorer is the 2D image resolution. We will see later that an inverse relationship exists between image resolution and frame rate."





\section{Wave Equations}
